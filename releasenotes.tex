\documentclass{article}      % Specifies the document class

%encoding
%--------------------------------------
\usepackage[ddmmyyyy]{datetime}
%--------------------------------------

%encoding
%--------------------------------------
\usepackage[utf8]{inputenc}
\usepackage[T1]{fontenc}
%--------------------------------------

%colors
%--------------------------------------
\usepackage[usenames, dvipsnames]{color}
%--------------------------------------

%code
%--------------------------------------
\usepackage{minted}
\definecolor{bg}{rgb}{0.95,0.95,0.95}
\newminted{python3}{bgcolor=bg, linenos=true, tabsize=4}

%-------------------------------------

%images
%--------------------------------------
\usepackage{graphicx}
%--------------------------------------

%Header and footer commands
%--------------------------------------
\usepackage{lastpage}
\usepackage{fancyhdr}
%--------------------------------------

%%%%%%%%%%%%%%%%%%%%%%%%%%%%%%%%%%%%%%%
%ALTERAR AQUI A VERSÃO DO DIGESTOR!!!!!
\newcommand{\versiondigester}{v1.0.1.1}
%%%%%%%%%%%%%%%%%%%%%%%%%%%%%%%%%%%%%%%

%Header and footer commands
%--------------------------------------
\pagestyle{fancy}
\fancyhf{}
\rhead{Digester \versiondigester}
\lhead{Yandeh}
\lfoot{Yandeh}
\rfoot{\today}
\fancyfoot[C]{\footnotesize Página \thepage\ de \pageref{LastPage}}

\fancypagestyle{firststyle}
{
   \fancyhf{}
   \rhead{Digester \versiondigester}
   \lhead{Yandeh}
   \lfoot{Yandeh}
   \cfoot{Página \thepage}
   \rfoot{\today}
   \fancyfoot[C]{\footnotesize Página \thepage\ de \pageref{LastPage}}
}

%Portuguese-specific commands
%--------------------------------------
\usepackage[portuguese]{babel}
%--------------------------------------

%Hyphenation rules
%--------------------------------------
\usepackage{hyphenat}
\hyphenation{mate-mática recu-perar}
%--------------------------------------

\title{Release Notes \\
      Digester \versiondigester}  % Declares the document's title.
\author{Yandeh Team}              % Declares the author's name.
\date{São Paulo, \today}       

\begin{document}             % End of preamble and beginning of text.

\maketitle                   % Produces the title.

\thispagestyle{firststyle}


\begin{table}[!ht]
\centering
\caption{Histórico de atualizações}
\label{my-label}
\begin{tabular}{|l|l|l|l|}
\hline
\textbf{Versão} & \textbf{Data} & \textbf{Descrição}                & \textbf{Autor}                                       \\ \hline
1.0.0.0           & 22/06/2016    & Revisão Inicial                 & Danilo Guanabara                                     \\ \hline
1.0.0.1           & 27/06/2016    & Bugfixes                        & Danilo Guanabara                                     \\ \hline
1.0.0.2           & 30/06/2016    & Bugfixes e limpeza              & Rosemary Sumitani                                    \\ \hline
1.0.0.3           & 04/07/2016    & Testes Unitários, bugfixes      & Danilo e Rosemary                                    \\ \hline
1.0.0.4           & 07/07/2016    & Parser Daruma, Bugfixes         & Luis, Danilo e Rosemary                              \\ \hline
1.0.0.5           & 17/07/2016    & Bugfixes Parser Daruma          & Luis Sant'Ana                                        \\ \hline
1.0.0.6           & 14/07/2016    & BugFixes Parser Daruma          & Luis Sant'Ana                                        \\ \hline  
1.0.0.7           & 18/07/2016    & BugFixes Troca Impressoras      & Luis e Danilo                                        \\ \hline
1.0.0.8           & 29/07/2016    & Bugfixes                        & Yandeh Team                                          \\ \hline
1.0.0.9           & 01/08/2016    & Parser SAT                      & Luis Sant'Ana                                        \\ \hline
1.0.1.0           & 04/08/2016    & Bugfixes                        & Luis Sant'Ana                                        \\ \hline
1.0.1.1           & 30/08/2616    & Arquivos truncados e aglutinados              & Luis Sant'Ana                                        \\ \hline        
\end{tabular}
\end{table}



\section{Objetivos}

Este documento tem o objetivo de descrever as alterações nas funcionalidades do Digester para a versão \versiondigester. 


\section{Introdução}     
Nesta versão foi adicionado o pré-processamento dos arquivos truncados da impressora Epson e os arquivos Daruma aglutinados. Devido às versões antigas do Coletor alguns arquivos estão truncados no repositório da Amazon, desta forma para não haver a total perda de informações relativas às vendas desses arquivos, foi desenvolvida uma solução parcial que atua no grupo de CNPJs com possibilidade de arquivos truncados. A outra solução implementada soluciona a questão de aglutinação dos comandos Daruma, gerados pela versão 3.0.1.8 do software Coletor EFC.

\section{O que mudou?}
Foram adicionados quatro scripts de processamento na linguagem de programação Perl, que são eles: $process\_threelines.pl$, $bema2epson.pl$, $precess\_daruma.pl$ e $trataDaruma.pl$, também foi alterado o \emph{script} em Python: $processa\_genericos.py$ para realizar as chamadas dos \emph{scripts} para cada arquivo de venda dos CNPJs truncados e/ou aglutinados. 

\section{Limitações}
As limitações se encontram em funcionalidades já destacadas em versões anteriores. Nesta versão não deve haver limitações quanto ao pré-processamento dos arquivos truncados e aglutinados.

\section{Como funciona?}

Dentro da estrutura de arquivos (/home/hadoop/Coleta/) existe um arquivo chamado $processa\_genericos.py$, 
no qual podemos selecionar o período de tempo que desejamos processar. Há duas possibilidades para realizar o processamento, através da função $process\_small\_generico\_month()$ apresentada no código numerado \ref{code:meses} onde podemos selecionar o intervalo de meses que desejamos processar, por exemplo, em no código \ref{code:meses} estamos processando dos meses Janeiro à Março (do mês 01 ao 03). Lembrando que o intervalo é semi-aberto $[a,b)$, ou seja, o valor $a$ está incluso e o valor $b$ não está incluso no intervalo. A outra possibilidade de processamento é através da função $process\_small\_generico\_day()$ que processa intervalos dentro de um mesmo mês, neste caso o intervalo é fechado $[a,b]$, onde os extremos a e b são inclusos no intervalo de processamento.    

\begin{listing}[H]
\begin{minted}{python}
def process_small_generico_month():
  mesInicial = 01
  mesFinal = 04
  ano = 2016
  for m in xrange(mesInicial, mesFinal):
    for d in xrange(01, calendar.monthrange(2016,m)[1] + 1):
       date_to_process = date(ano, m, d)
       print("INFO: Processando Data: %s" % date_to_process)
       process_each_cnpj_by_date(date_to_process, GENERICO_INPUT_DIR)
\end{minted}
\caption{Processamento de intervalo de meses}
\label{code:meses}
\end{listing}


\begin{listing}[H]
\begin{minted}{python}
def process_small_generico_day():
  diaInicial = 01
  diaFinal = 01
  mes = 06
  ano = 2016
  for d in xrange(diaInicial, diaFinal+1):
     date_to_process = date(ano, mes, d)
     print("INFO: Processando Data: %s" % date_to_process)
     process_each_cnpj_by_date(date_to_process, GENERICO_INPUT_DIR)
\end{minted}
\caption{Processamento de intervalo de dias}
\label{code:dias}
\end{listing}


Arquivos de logs com extensão \textbf{.log} serão gerados no diretório $Coleta$ para auxiliar em caso de problemas. Foram adicionados dois arquivos de log aos existentes no caso dos arquivos truncados, para checagem do pré-processamento, os arquivos: $process\_threelines.log$ e o arquivo $bema2epson.log$. Para os arquivos algutinados foram adicionados os arquivos $trata_daruma.log$ e $process_daruma.log$. 

O resultado do processamento do código $process\_threelines.pl$ pode ser visto no diretório $Genericos/pre$, que são lidos pelo \emph{script} $bema2epson.pl$, este por sua vez gera arquivos na pasta $Genericos/pos$. Da mesma forma o resultado do processamento gerado por $trataDaruma.pl$ pode ser visto na pasta $Genericos/trata_daruma$ e o resultado de $process_daruma.pl$ pode ser visto em $Genericos/process_daruma$.

O fluxo completo consiste em gerar os arquivos concatenados de um dia, na pasta $Cupons/truncate$, que são processados por $process\_threelines.pl$ e depois processados por $bema2epson.pl$. Por fim, são copiados os arquivos para a pasta \textbf{Cupons/processar} no qual são passados para o digestor processar. De forma similar, o mesmo ocorre com o processo de aglutinação dos arquivos Daruma, alterando os scripts de excução e as pastas temporárias.

O programa digere os dados da pasta \textbf{Cupons/processar (.NS)} e o resultado da digesão dos dados estarão na pasta \textbf{Cupons/processados (.csv, .pag).}.  \footnote{Obs. Sempre que possível cheque se a pasta a ser processada contém os arquivos coletados.}.

No código apresentado em \ref{code:processamento} a chamada à função $truncate\_conc\_generico()$ caso seja comentada, não realizará o pre-processamento definido nas funções de intervalos de tempo, e os CNPJs definidos no dicionário $CNPJ\_Exclude$ não serão processados. De forma análago, a função     $aglutinate\_conc\_generico()$ realiza o processamento dos CNPJs inclusos no dicionário $CNPJ_daruma$.
 
\begin{listing}[H]
\begin{minted}{python}
def main():
    sync()
    
    #process_small_generico_month()
    process_small_generico_day()

    #date_to_process = date.today() - timedelta(days=1)
    
    #print("INFO: Processando Data: %s" % date_to_process)
    
    #process_each_cnpj_by_date(date_to_process, GENERICO_INPUT_DIR)

    sync()
    truncate_conc_generico()
    aglutinate_conc_generico()
    process_conc_generico()
\end{minted}

\caption{Adição do pré-processamento}
\label{code:processamento}

\end{listing}


A lista de CNPJs identificados como truncados está apresentada a seguir:

\begin{enumerate}
\item 14152893000103 
\item 15352063000184 
\item 16714499000139
\item 08976226000134
\item 12532740000158 
\item 12928795000181 
\item 14954187000177 
\item 60394343000100 
\item 04949587000130 
\item 11223033000117
\item 11237532000163
\item 16675912000101 
\item 60740867000105 
\item 10805439000145
\item 12998624000129 
\item 11122233344455 
\item 12345678901234
\item 12426656000102
\item 18065345000199
\item 23343955000147
\item 77777777777777
\end{enumerate}


As maiores frequências de truncamento se devem ao fato de um comando de uma impressora Epson, que apresenta o formato: $\#ACK\#STX\#\{...\}$ vir sem o prefixo $\#ACK$ e se transformar em um comando do tipo Bematech, no formato: $\#STX\#AAA\#AAA\#ESC\{...\}$, onde $\#AAA$ pode ser qualquer comando. Outra frequência alta são quebras de linhas em descrições de itens de venda. Neste caso são perdidos informações relativas ao item, o que pode ocasionar a impossibilidade de carga no banco de dados, através da execução do ETL \emph{(Extract, Transformation and Load)}. Outros casos podem ser identificados, o que acarretará a adição de tratamento para esse novos episódios.  

Em um próximo passo espera-se corrigir os arquivos em definitivo em vez da abordagem utilizando o pré-processamento, neste caso serão reescritos os arquivos da forma correta e realizada a substituição dos dos mesmos no balde do serviço S3 da Amazon. 

Para os arquivos aglutinados apenas há uma constatação de aglutinação dos comandos, encontrados no CNPJ: \textbf{02523765000131}.


\section{Sugestão de testes}

\begin{itemize}
    \item Reprocessamento dos meses com arquivos truncados e aglutinados.  
    \item Carga dos dados no banco de dados através do ETL.
    \item Validação manual dos arquivos .NS gerados na pasta $Genericos/pos$ e $Genericos/process\_daruma$
\end{itemize}



%\begin{thebibliography}{1}
%  \bibitem{sat} {\em Especificação Técnica de Requisitos - SAT. pag. 96}  2016.
%\end{thebibliography}

\end{document}               % End of document.