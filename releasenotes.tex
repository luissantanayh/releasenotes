\documentclass{article}      % Specifies the document class

%encoding
%--------------------------------------
\usepackage[ddmmyyyy]{datetime}
%--------------------------------------

%encoding
%--------------------------------------
\usepackage[utf8]{inputenc}
\usepackage[T1]{fontenc}
%--------------------------------------

%colors
%--------------------------------------
\usepackage[usenames, dvipsnames]{color}
%--------------------------------------

%code
%--------------------------------------
\usepackage{minted}
\definecolor{bg}{rgb}{0.95,0.95,0.95}
\newminted{python3}{bgcolor=bg, linenos=true, tabsize=4}

%-------------------------------------

%images
%--------------------------------------
\usepackage{graphicx}
%--------------------------------------

%Header and footer commands
%--------------------------------------
\usepackage{lastpage}
\usepackage{fancyhdr}
%--------------------------------------

%%%%%%%%%%%%%%%%%%%%%%%%%%%%%%%%%%%%%%%
%ALTERAR AQUI A VERSÃO DO DIGESTOR!!!!!
\newcommand{\versiondigester}{v1.0.1.2}
%%%%%%%%%%%%%%%%%%%%%%%%%%%%%%%%%%%%%%%

%Header and footer commands
%--------------------------------------
\pagestyle{fancy}
\fancyhf{}
\rhead{Digester \versiondigester}
\lhead{Yandeh}
\lfoot{Yandeh}
\rfoot{\today}
\fancyfoot[C]{\footnotesize Página \thepage\ de \pageref{LastPage}}

\fancypagestyle{firststyle}
{
   \fancyhf{}
   \rhead{Digester \versiondigester}
   \lhead{Yandeh}
   \lfoot{Yandeh}
   \cfoot{Página \thepage}
   \rfoot{\today}
   \fancyfoot[C]{\footnotesize Página \thepage\ de \pageref{LastPage}}
}

%Portuguese-specific commands
%--------------------------------------
\usepackage[portuguese]{babel}
%--------------------------------------

%Hyphenation rules
%--------------------------------------
\usepackage{hyphenat}
\hyphenation{mate-mática recu-perar}
%--------------------------------------

\title{Release Notes \\
      Digester \versiondigester}  % Declares the document's title.
\author{Yandeh Team}              % Declares the author's name.
\date{São Paulo, \today}       

\begin{document}             % End of preamble and beginning of text.

\maketitle                   % Produces the title.

\thispagestyle{firststyle}


\begin{table}[!ht]
\centering
\caption{Histórico de atualizações}
\label{my-label}
\begin{tabular}{|l|l|l|l|}
\hline
\textbf{Versão} & \textbf{Data} & \textbf{Descrição}                & \textbf{Autor}                                       \\ \hline
1.0.0.0           & 22/06/2016    & Revisão Inicial                 & Danilo Guanabara                                     \\ \hline
1.0.0.1           & 27/06/2016    & Bugfixes                        & Danilo Guanabara                                     \\ \hline
1.0.0.2           & 30/06/2016    & Bugfixes e limpeza              & Rosemary Sumitani                                    \\ \hline
1.0.0.3           & 04/07/2016    & Testes Unitários, bugfixes      & Danilo e Rosemary                                    \\ \hline
1.0.0.4           & 07/07/2016    & Parser Daruma, Bugfixes         & Luis, Danilo e Rosemary                              \\ \hline
1.0.0.5           & 17/07/2016    & Bugfixes Parser Daruma          & Luis Sant'Ana                                        \\ \hline
1.0.0.6           & 14/07/2016    & BugFixes Parser Daruma          & Luis Sant'Ana                                        \\ \hline  
1.0.0.7           & 18/07/2016    & BugFixes Troca Impressoras      & Luis e Danilo                                        \\ \hline
1.0.0.8           & 29/07/2016    & Bugfixes                        & Yandeh Team                                          \\ \hline
1.0.0.9           & 01/08/2016    & Parser SAT                      & Luis Sant'Ana                                        \\ \hline
1.0.1.0           & 04/08/2016    & Bugfixes                        & Luis Sant'Ana                                        \\ \hline
1.0.1.1           & 30/08/2616    & Arquivos truncados e aglutinados              & Luis Sant'Ana                          \\ \hline
1.0.1.2           & 05/09/2616    & Bugfixes                        & Luis Sant'Ana                                         \\ \hline
\end{tabular}
\end{table}



\section{Objetivos}

Este documento tem o objetivo de corrigir bugs do Digester para a versão anterior v1.0.1.1. 


\section{Introdução}     
Nesta versão foram corrigidos \emph{bugs} relativos a funcionalidade de truncamento de arquivos Epson. A funcionalidade de correção de arquivos aglutinados Daruma foi removida, devido a complexidade relacionada a solução de todos os possíveis casos, o que acarreta esforço demasiado, o que não compensa devido à baixa incidência.

\section{O que mudou?}
Foram removidos os \emph{scripts} relacionados ao processsamento de arquivos Daruma e foram corrigidos \emph{bugs} encontrados, relacionados aos arquivos truncados.

\section{Lista de Bugs encontrados}
\subsection{Registro de inconsistências}
\begin{enumerate}
    \item \emph{Primeira linha do registro foi registrada de forma não esperada:} Este tipo de situação é comum e esperada nos arquivos \textbf{.NS}, e não traz prejuízo na captura de informações pelo digestor, devido a quebra dos arquivos em dias. 
    \item \emph{As últimas linhas foram finalizadas de forma não esperada:} Da mesma forma que o item anterior este tipo de caso é comum e não tem impacto sobre a capacidade de processamento do digestor.
\end{enumerate}

\subsection{\emph{Issues}}
\begin{enumerate}
    \item \emph{Nome do arquivo não apresenta a extenção .NS}. Questão não identificada, o arquivo em questão está com a extensão na pasta \emph{pos}. Além de este caso não ter impacto no processamento do digestor.
    \item \emph{Vir um número zero em resposta a um comando:} Este tipo de resposta é comum e não tem impacto sobre o processamento.
    \item \emph{Início da linha com números:} Resposta comum da impressora, sem prejuízo de informações.
    \item \emph{Truncamento de rodapé de cupom:} Tratamento realizado através da modificação da regex para capturar a linha.
    \item \emph{Início da linha com números:} Resposta comum da impressora, sem prejuízo de informações.
    \item \emph{Quebra de linha em comando $\#ACK$}: Tratamento de realizado através da modificação de regex.
    \item \emph{Outros casos não listados}: vários outros casos foram identificados, o que não foi listado aqui, pois se enquadra em um dos casos anteriores.
\end{enumerate}


\section{Limitações}
As limitações se encontram em funcionalidades já destacadas em versões anteriores. Nesta versão não deve haver limitações quanto ao pré-processamento dos arquivos truncados Epson.

\section{Como funciona?}

Dentro da estrutura de arquivos (/home/hadoop/Coleta/) existe um arquivo chamado $processa\_genericos.py$, 
no qual podemos selecionar o período de tempo que desejamos processar. Há duas possibilidades para realizar o processamento, através da função $process\_small\_generico\_month()$ apresentada no código numerado \ref{code:meses} onde podemos selecionar o intervalo de meses que desejamos processar, por exemplo, em no código \ref{code:meses} estamos processando dos meses Janeiro à Março (do mês 01 ao 03). Lembrando que o intervalo é semi-aberto $[a,b)$, ou seja, o valor $a$ está incluso e o valor $b$ não está incluso no intervalo. A outra possibilidade de processamento é através da função $process\_small\_generico\_day()$ que processa intervalos dentro de um mesmo mês, neste caso o intervalo é fechado $[a,b]$, onde os extremos a e b são inclusos no intervalo de processamento.    

\begin{listing}[H]
\begin{minted}{python}
def process_small_generico_month():
  mesInicial = 01
  mesFinal = 04
  ano = 2016
  for m in xrange(mesInicial, mesFinal):
    for d in xrange(01, calendar.monthrange(2016,m)[1] + 1):
       date_to_process = date(ano, m, d)
       print("INFO: Processando Data: %s" % date_to_process)
       process_each_cnpj_by_date(date_to_process, GENERICO_INPUT_DIR)
\end{minted}
\caption{Processamento de intervalo de meses}
\label{code:meses}
\end{listing}


\begin{listing}[H]
\begin{minted}{python}
def process_small_generico_day():
  diaInicial = 01
  diaFinal = 01
  mes = 06
  ano = 2016
  for d in xrange(diaInicial, diaFinal+1):
     date_to_process = date(ano, mes, d)
     print("INFO: Processando Data: %s" % date_to_process)
     process_each_cnpj_by_date(date_to_process, GENERICO_INPUT_DIR)
\end{minted}
\caption{Processamento de intervalo de dias}
\label{code:dias}
\end{listing}


Arquivos de logs com extensão \textbf{.log} serão gerados no diretório $Coleta$ para auxiliar em caso de problemas. Foram adicionados dois arquivos de log aos existentes no caso dos arquivos truncados, para checagem do pré-processamento, os arquivos: $process\_threelines.log$ e o arquivo $bema2epson.log$. Nesta versão, as linhas modificadas são escritas nos arquivos de $log$ em formato JSON. 

O resultado do processamento do código $process\_threelines.pl$ pode ser visto no diretório $Genericos/pre$, que são lidos pelo \emph{script} $bema2epson.pl$, este por sua vez gera arquivos na pasta $Genericos/pos$. 

O fluxo completo consiste em gerar os arquivos concatenados de um dia, na pasta $Cupons/truncate$, que são processados por $process\_threelines.pl$ e depois processados por $bema2epson.pl$. Por fim, são copiados os arquivos para a pasta \textbf{Cupons/processar} no qual são passados para o digestor processar. 

O programa digere os dados da pasta \textbf{Cupons/processar (.NS)} e o resultado da digesão dos dados estarão na pasta \textbf{Output (.csv, .pag).}.  \footnote{Obs. Sempre que possível cheque se a pasta a ser processada contém os arquivos coletados.}.

No código apresentado em \ref{code:processamento} a chamada à função $truncate\_conc\_generico()$ caso seja comentada, não realizará o pre-processamento definido nas funções de intervalos de tempo, e os CNPJs definidos no dicionário $CNPJ\_Exclude$ não serão processados.
 
\begin{listing}[H]
\begin{minted}{python}
def main():
    sync()
    
    #process_small_generico_month()
    process_small_generico_day()

    #date_to_process = date.today() - timedelta(days=1)
    
    #print("INFO: Processando Data: %s" % date_to_process)
    
    #process_each_cnpj_by_date(date_to_process, GENERICO_INPUT_DIR)

    sync()
    truncate_conc_generico()
    process_conc_generico()
\end{minted}

\caption{Adição do pré-processamento}
\label{code:processamento}

\end{listing}


A lista de CNPJs identificados como truncados está apresentada a seguir:

\begin{enumerate}
\item 14152893000103 
\item 15352063000184 
\item 16714499000139
\item 08976226000134
\item 12532740000158 
\item 12928795000181 
\item 14954187000177 
\item 60394343000100 
\item 04949587000130 
\item 11223033000117
\item 11237532000163
\item 16675912000101 
\item 60740867000105 
\item 10805439000145
\item 12998624000129 
\item 11122233344455 
\item 12345678901234
\item 12426656000102
\item 18065345000199
\item 23343955000147
\item 77777777777777
\end{enumerate}


As maiores frequências de truncamento se devem ao fato de um comando de uma impressora Epson, que apresenta o formato: $\#ACK\#STX\#\{...\}$ vir sem o prefixo $\#ACK$ e se transformar em um comando do tipo Bematech, no formato: $\#STX\#AAA\#AAA\#ESC\{...\}$, onde $\#AAA$ pode ser qualquer comando. Outra frequência alta são quebras de linhas em descrições de itens de venda. Neste caso são perdidos informações relativas ao item, o que pode ocasionar a impossibilidade de carga no banco de dados, através da execução do ETL \emph{(Extract, Transformation and Load)}. Outros casos podem ser identificados, o que acarretará a adição de tratamento para esse novos episódios.  

Em um próximo passo espera-se corrigir os arquivos em definitivo em vez da abordagem utilizando o pré-processamento, neste caso serão reescritos os arquivos da forma correta e realizada a substituição dos dos mesmos no balde do serviço S3 da Amazon. 


\section{Sugestão de testes}

\begin{itemize}
    \item Reprocessamento dos meses com arquivos truncados. 
    \item Carga dos dados no banco de dados através do ETL.
    \item Validação manual dos arquivos .NS gerados na pasta $Genericos/pos$.
\end{itemize}



%\begin{thebibliography}{1}
%  \bibitem{sat} {\em Especificação Técnica de Requisitos - SAT. pag. 96}  2016.
%\end{thebibliography}

\end{document}               % End of document.